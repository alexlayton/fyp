
%!TEX TS-program = xelatex
%!TEX encoding = UTF-8 Unicode

\documentclass[12pt]{article}
\usepackage{geometry}
\geometry{a4paper}   
\usepackage{graphicx}
\usepackage{amssymb}
\usepackage{tikz}
\usepackage{gantt}
\usepackage{fontspec,xltxtra,xunicode}
\defaultfontfeatures{Mapping=tex-text}
\setromanfont[Mapping=tex-text]{Gill Sans}

\title{{\small Final Year Project Proposal:} \\ A Location Aware Reminder App for iOS}
\author{Alex Layton \\ {\small Supervised by Russell Beale}}
\date{19 April, 2012} 

\begin{document}
\maketitle

\begin{abstract}
Smartphones have become an every day essential item for most people, with more and more using their phones to help them with tasks such as calendaring and setting reminders. I propose an idea for an app somewhere between these two things. Apple's iOS was recently updated to include a Reminders app that allows a reminder to be triggered from a specified location. My idea is to create an app that can trigger a reminder at a time, dependant on how far away the user is from their destination, specifically for students.     
\end{abstract}

\section*{Description}
In this project I wish to create an app for Apple's iOS mobile operating system. The app will be primarily aimed a students, but I hope to create a framework of sorts that can be easily used in other apps to allow for similar location aware reminders.\\ The app will need to connect to a college's database server in order to download a students timetable. Once this has been done the student will be able to see their timetable on their device, as well as select which items they wish to receive location aware reminders for. The GPS in the students device will be used to check how far away from a lecture they are and remind them about their lecture a certain amount of time before, dependant on the students current location. For instance, if the student is 5 minutes away from their lectures. A reminder will be sent at least 5 minutes before the lecture time to tell the student they should head to lecture now.\\ The app will make use of several features built into iOS, such as Core Location which will allow me to use the location date from the users current position. I will also make use of UIMapView framework to allow me to show the users position on a map. The most import feature is the Significant Change Service which runs in the background on iOS devices. By using the service the app will be able to run in the background for a short time when the user's location changes. In this time I can check to see where they are in relation to the target location and send a Push Notification to remind the user if necessary.

\section*{Approach}
How I approach the project will be dependant on the modules I decide on taking in my final year. I hope to have the basic functionality implemented before the end of the year. This will then allow me to add the extensions and polish to the app during the second term. When creating the app their will be quite a large focus on the actual user interface. I should spend the same amount of time on the design of the app as I do the implementation, but in practice it will be more likely be half the development time. Once the app has been completed I will work on the presentation and report.

\section*{Extensions}
Due to the fact that I wish to create an iOS app, an obvious extension would be making the app universal. This would mean that the same app binary would work on both Apple's iPhone and iPad devices. This would also mean I would have to create two separate user interfaces specific to each device. Since this app is aimed primarily at students, I could also include a timetable view of their lectures allowing them to see what that are doing all week and not just on one day.

\section*{Feasibility}
To create an iOS app there are some prerequisites, these include an OS X based machine that can run Xcode - Apple's IDE for Objective-C apps and devices running iOS which can be used for testing. I have both of these things, so building the application will not require me purchasing any hardware. If I was to release the app on Apple's App Store, I would need a developers license. This is something I plan to buy independently, but is not required to initially build the app.\\ I feel that the idea I have proposed above is very feasible in terms of resources and the time frame given, but if I am not able to fully implement my the app a fallback could be an implementation of the location based reminders functionality packaged as a standalone app which would have reminders set manually as opposed to creating them from a student's timetable. 

\section*{Timeplan}
	
  \begin{gantt}{10}{12}
    \begin{ganttitle}
    \numtitle{1}{1}{12}{1}
    \end{ganttitle}
    
    \ganttbar{Interface design}{0}{1}
    \ganttbarcon{Interface in Xcode}{1}{2}
    
    \ganttbar{Core Location}{1}{3}
    \ganttbarcon{User Login}{3}{1}
    \ganttbarcon{Timetable integration}{4}{1}
    \ganttbarcon{Push Notifications}{5}{1}
    
    \ganttbar{Project Presentation}{8}{1}
    \ganttbarcon{Project Report}{9}{3}
    
    \ganttbar{Extensions}{6}{2}
  \end{gantt}

\end{document}  